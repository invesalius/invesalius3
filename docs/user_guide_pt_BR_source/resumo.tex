\newpage
\vspace*{10pt}
\thispagestyle{empty}

\begin{center} \emph{\begin{large}  Sobre o InVesalius \end{large}}
\vspace{2pt}
\end{center}

\onehalfspacing
 		
InVesalius é um software público para a área de saúde que realiza análise e segmentação de
modelos anatômicos virtuais, possibilitando a confecção de modelos físicos com o auxílio da
prototipagem rápida.
A partir de imagens em duas dimensões (2D) obtidas por meio de equipamentos de Tomografia
Computadorizada (TC) ou Ressonância Magnética (RM), o programa permite criar modelos
virtuais em três dimensões (3D) correspondentes às estruturas anatômicas dos pacientes em
acompanhamento médico.

O nome InVesalius é uma homenagem ao médico belga Andreas Vesalius (1514-1564),
considerado o "pai da anatomia moderna".
O software InVesalius é desenvolvido pelo CTI (Centro de Tecnologia da Informação Renato
Archer), unidade do Ministério da Ciência e Tecnologia (MCT), desde 2001. Inicialmente, apenas
o programa de instalação era distribuído gratuitamente. A partir de novembro de 2007,
o InVesalius foi disponibilizado como software livre no Portal do Software Público, 
consolidando comunidades de usuários e de desenvolvedores.
Trata-se de uma ferramenta simples, livre e gratuita,
robusta, multiplataforma, com comandos em Português, com funções claras e diretas, de fácil
manuseio e rápida quando executada em microcomputador PC.

O uso das tecnologias de visualização e análise tridimensional de imagens médicas, integradas
ou não a prototipagem rápida, auxiliam o cirurgião no diagnóstico de patologias e permitem que
seja realizado um planejamento cirúrgico detalhado, simulando com antecedência intervenções
complexas, que podem envolver, por exemplo, alto grau de deformidade facial ou a colocação
de próteses.

O InVesalius tem demonstrado grande versatilidade e vem contribuindo com diversas áreas,
dentre as quais medicina, odontologia, veterinária, arqueologia e engenharia.

\newpage

Opções para download:

\begin{itemize}
	\item \href{https://www.cti.gov.br/invesalius}{https://www.cti.gov.br/invesalius}
	\item \href{http://invesalius.github.io}{http://invesalius.github.io}	
	\item \href{http://www.softwarepublico.gov.br}{www.softwarepublico.gov.br}
\end{itemize}

\noindent
