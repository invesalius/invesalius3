\newpage
\vspace*{10pt}
\thispagestyle{empty}

\begin{center} \emph{\begin{large}  About InVesalius \end{large}}
\vspace{2pt}
\end{center}

\onehalfspacing
 		
InVesalius is public health software that performs analysis and segmentation of
Virtual anatomical models, enabling the creation of physical models with the aid of
rapid prototyping (3D printing).
From two-dimensional (2D) images obtained through Computed Tomography (CT) or Magnetic Resonance Imaging (MRI), the program allows users to create
three-dimensional (3D) anatomical representations of patients for further medical use.

InVesalius is named in honour of the Belgian doctor Andreas Vesalius (1514-1564), widely
considered the father of modern anatomy.
InVesalius software is developed by CTI (Center for Information Technology Center Renato
Archer), a unit of the Brazilian Ministry of Science and Technology (MCT), since 2001. Initially, only
the installation program was distributed as freeware. In November 2007
InVesalius was made fully available as free software and open source via the Public Software Portal, allowing for communities of users and developers to connect.
InVesalius is, in short, a simple yet robust, free cross-platform tool that is easy to use.


The use of visualization technologies and three-dimensional analysis of medical images, when combined with rapid prototyping (3D printing), assists the surgeon in diagnosing pathologies and a detailed surgical planning and simulation of complex process in advance, which may involve, for example, a high degree of facial deformity or integration of prosthetics.

InVesalius has demonstrated great versatility and has applications in different areas, including medicine, dentistry, veterinary medicine, archeology and engineering.

Download options:

\begin{itemize}
	\item \href{https://www.cti.gov.br/invesalius}{https://www.cti.gov.br/invesalius}
	\item \href{http://invesalius.github.io}{http://invesalius.github.io}	
	\item \href{http://www.softwarepublico.gov.br}{www.softwarepublico.gov.br}
\end{itemize}
		
\noindent